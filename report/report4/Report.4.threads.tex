Explain how you improve the labwork:

__global__ void grayScale2D(uchar3 *input, uchar3 *output, int widthImage, int heightImage){
    int width = blockDim.x * gridDim.x;
    
    int tidX = threadIdx.x + blockIdx.x * blockDim.x;
    int tidY = threadIdx.y + blockIdx.y * blockDim.y;

    if (tidY * width + tidX < (widthImage * heightImage)) {
        output[tidY * width + tidX].x = (input[tidY * width + tidX].x + 
                                        input[tidY * width + tidX].y +
                                        input[tidY * width + tidX].z) / 3;
        output[tidY * width + tidX].z = output[tidY * width + tidX].y = output[tidY * width + tidX].x;       
    }
}

void Labwork::labwork4_GPU() {
    // Calculate number of pixels
    int pixelCount = inputImage->width * inputImage->height;
    
    // Allocate CUDA memory
    uchar3 *d_inputImage;
    uchar3 *d_grayImage;
    cudaMalloc(&d_inputImage, pixelCount * 3);
    cudaMalloc(&d_grayImage, pixelCount * 3);

    // Copy CUDA Memory from CPU to GPU
    cudaMemcpy(d_inputImage, inputImage->buffer, pixelCount * 3, cudaMemcpyHostToDevice);

    // Processing
    dim3 blockSize = dim3(32,32);
    dim3 gridSize = dim3(inputImage->width/blockSize.x, inputImage->height/blockSize.y);
    
    if (inputImage->width % blockSize.x != 0){
        gridSize.x += 1;
    }

    if (inputImage->height % blockSize.y != 0){
        gridSize.y += 1;
    }

    Timer t;
    t.start();
    grayScale2D<<<gridSize, blockSize>>>(d_inputImage, d_grayImage, inputImage->width, inputImage->height);
    cudaDeviceSynchronize();

    printf("time elapsed : %fms\n", t.getElapsedTimeInMilliSec());
    // Copy CUDA Memory from GPU to CPU
    uchar3 *outputGrayImage;
    outputGrayImage = (uchar3 *) malloc(pixelCount * 3);
    cudaMemcpy(outputGrayImage, d_grayImage, pixelCount * 3, cudaMemcpyDeviceToHost);

    // Cleaning
    cudaFree(d_grayImage);
    cudaFree(d_inputImage);
    outputImage = (char*) outputGrayImage;
}
